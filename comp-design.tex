
\documentclass[letterpaper,11pt]{article}

\usepackage{amsmath}

\pagenumbering{gobble}

\begin{document}
{
	\title{CS 6800: Comprehensive Design Document}
	\author{Alex Thannhauser}
	\date{December 13th 2021}

	\maketitle
	
	\section{Survey of the Design}
	{
		\subsection{Scanner and New Grammar Structure}
		{
			The original grammar had to be rearranged to support operator
			precedence and to remove the shift/reduce errors. The grammar rules
			for $\langle letter \rangle$, $\langle digit \rangle$,
			$\langle name \rangle$, and $\langle numeral \rangle$ have been made
			into tokens. Below is the new structure for the grammar:
			
			\begin{align*}
				\langle \textit{primary\_expression} \rangle =
					& \text{ NAME }\\
					& \cup \text{ NUMERAL }\\
					& \cup ( \langle expression \rangle )\\
				\langle \textit{prefix\_expression} \rangle =
					& \langle \textit{primary\_expression} \rangle\\
					& \cup NOT \langle \textit{primary\_expression} \rangle\\
				\langle \textit{multiplicative\_expression} \rangle =
					& \langle \textit{prefix\_expression} \rangle\\
					& \cup \langle \textit{multiplicative\_expression} \rangle \times \langle \textit{prefix\_expression} \rangle\\
					& \cup \langle \textit{multiplicative\_expression} \rangle / \langle \textit{prefix\_expression} \rangle\\
				\langle \textit{additive\_expression} \rangle =
					& \langle \textit{multiplicative\_expression} \rangle\\
					& \cup \langle \textit{additive\_expression} \rangle + \langle \textit{multiplicative\_expression} \rangle\\
					& \cup \langle \textit{additive\_expression} \rangle - \langle \textit{multiplicative\_expression} \rangle\\
				\langle \textit{relational\_expression} \rangle =
					& \langle \textit{additive\_expression} \rangle\\
					& \cup \langle \textit{additive\_expression} \rangle < \langle \textit{additive\_expression} \rangle\\
					& \cup \langle \textit{additive\_expression} \rangle > \langle \textit{additive\_expression} \rangle\\
					& \cup \langle \textit{additive\_expression} \rangle \leq \langle \textit{additive\_expression} \rangle\\
					& \cup \langle \textit{additive\_expression} \rangle \geq \langle \textit{additive\_expression} \rangle\\
				\langle \textit{equality\_expression} \rangle =
					& \langle \textit{relational\_expression} \rangle\\
					& \cup \langle \textit{relational\_expression} \rangle = \langle \textit{relational\_expression} \rangle\\
					& \cup \langle \textit{relational\_expression} \rangle \neq \langle \textit{relational\_expression} \rangle\\
				\langle \textit{logical\_and\_expression} \rangle =
					& \langle \textit{equality\_expression} \rangle\\
					& \cup \langle \textit{equality\_expression} \rangle \land \langle \textit{equality\_expression} \rangle\\
				\langle \textit{logical\_or\_expression} \rangle =
					& \langle \textit{logical\_and\_expression} \rangle\\
					& \cup \langle \textit{logical\_and\_expression} \rangle \lor \langle \textit{logical\_and\_expression} \rangle\\
				\langle \textit{expression} \rangle = & \langle \textit{logical\_or\_expression} \rangle\\
				\langle \textit{maybe\_else} \rangle =
					& \cup \text{ ELSE } \langle \textit{statement\_plus} \rangle\\
				\langle \textit{statement} \rangle =
					& \text{ NAME } \text{ COLON } \langle \textit{statement} \rangle\\
					& \cup \text{ NAME } \text{ ASSIGN } \langle \textit{expression} \rangle\\
					& \cup \text{ GOTO } \text{ NAME } \\
					& \cup \text{ STOP } \\
					& \cup \text{ LOOP } \langle \textit{expression} \rangle \text{ DO } \langle \textit{statement\_plus} \rangle \text{ END }\\
					& \cup \text{ WHILE } \langle \textit{expression} \rangle \text{ DO } \langle \textit{statement\_plus} \rangle \text{ END }\\
					& \cup \text{ IF } \langle \textit{expression} \rangle \text{ THEN }
						\langle \textit{statement\_plus} \rangle \langle \textit{maybe\_else} \rangle \text{ END }\\
				\langle \textit{statement\_plus} \rangle =
					& \langle \textit{statement} \rangle \\
					& \cup \langle \textit{statement} \rangle \langle \textit{statement\_plus} \rangle\\
				\langle \textit{program} \rangle = &\langle \textit{statement\_plus} \rangle\\
			\end{align*}
			
		}
		
		\subsection{Program Structure}
		{
			Below is the overall structure of the program:
			\begin{enumerate}
			{
				\item Parse command-line flags
				\item Allocate variable and label scope. (\texttt{avl\_alloc\_tree})
				\item Add each variable declared in the command-line flags into
					variable scope (\texttt{avl\_insert})
				\item Open and Parse Input File, yielding a statement graph.
				\item \texttt{statement = } start
				\item \texttt{while statement != NULL}:
				\begin{enumerate}
					\item Execute statement, yielding the next statement to execute:
						\begin{itemize}
							\item assignment statement:
								Evaluate the expression tree and either declare
								(\texttt{avl\_insert}) or assign (\texttt{avl\_search})
								the named variable's value.
							\item goto statement:
								set \texttt{next statement} to the statement
								associated with the given label.
							\item if-else statement:
								evaluate expression-tree conditional. If the
								result is a non-zero value, set
								\texttt{next statement} to refer to the first
								statement of the true case, otherwise: false
								case.
							\item while statement:
								The reduction rules for the while statement
								create a if-else statement where the 'true-case'
								would refer to the body of the while loop, with
								an added GOTO back to the conditional added to
								the end, and where the 'false-case' refers to
								the first statement after the while loop.
							\item loop statement:
								if it is the first time executing this statement,
								evaluate expression tree and save result, otherwise:
								use already saved value.
								if value is nonzero, set \texttt{next statement}
								to the first statement of the body and decrement
								value, otherwise: the first statement after the
								loop and consider next time executing as
								"first time".
								The reduction rules that generated the loop
								statement add a GOTO back to the loop statement
								at the end of the body.
						\end{itemize}
					\item set current statement to \texttt{next statement}
				\end{enumerate}
				\item Print all variables in scope
				\item Free statement graph
				\item Free scopes
				\item Free command-line flags
			}
			\end{enumerate}
		}
	}

}
\end{document}






















