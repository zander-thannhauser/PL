
\documentclass[letterpaper,11pt]{article}

\pagenumbering{gobble}

\begin{document}
{
	\title{CS 6800: Preliminary Design Document}
	\author{Alex Thannhauser}
	\date{December 13th 2021}

	\maketitle
	
	\section{Possible Approaches}
	{
		\begin{itemize}
		{
			\item Use Bison and Flex for generating the parser.
			\item Use GMP for integer computation.
			\item Use Reference Counting for determining when to call
				\texttt{free} on \texttt{malloc}-ed memory.
			\item Use Make and GCC for building and compiling the program. Make
				is also used for building GMP if not already installed in
				the system.
			\item Ensure that at the execution of the program there are no
				memory leaks and that there are no valgrind errors.
		}
		\end{itemize}
	}
	
	\section{Research So Far}
	{
		\begin{itemize}
		{
			\item \texttt{mpz\_add}, \texttt{mpz\_sub}, \texttt{mpz\_mul}, and
				\texttt{mpz\_fdiv\_q} are the functions in the GMP library for
				doing additiona, subtraction, multiplication and division
				respectively.
			\item \texttt{libavl} can be used to create a binary tree to map
				variable names to values.
				(\texttt{avl\_alloc\_tree} \& \texttt{avl\_search}).
		}
		\end{itemize}
	}
	
	\section{Design Flow}
	{
		Below is the overall structure of the program:
		\begin{enumerate}
		{
			\item Parse command-line flags
			\item Allocate variable and label scope. \texttt{avl\_alloc\_tree}
			\item Add each variable declared in the command-line flags into
				variable scope (\texttt{avl\_insert})
			\item Open and Parse Input File, yielding a statement graph.
			\item \texttt{statement = } start
			\item \texttt{while statement != NULL}:
			\begin{enumerate}
				\item Execute statement, yielding the next statement to execute.
				\item \texttt{statement = } next statement
			\end{enumerate}
			\item Print all variables in scope
			\item Free statement graph
			\item Free scopes
			\item Free command-line flags
		}
		\end{enumerate}
	}

}
\end{document}






















