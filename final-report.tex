
\documentclass[letterpaper,11pt]{article}

\pagenumbering{gobble}

\begin{document}
{
	\title{CS 6800: Final Report}
	\author{Alex Thannhauser}
	\date{December 13th 2021}

	\maketitle
	
	\section{Project Description}
	{
		The main goal of this project is to write an implmentation for the
		extended PL language. The program would be able to read and execute
		a program described in an input file and would print the final state of
		all variables involved in the computation.
	}
	
	\section{Research/Design}
	{
		Below is the overall structure of the program:
		\begin{enumerate}
		{
			\item Parse command-line flags
			\item Allocate variable and label scope. \texttt{avl\_alloc\_tree}
			\item Add each variable declared in the command-line flags into
				variable scope (\texttt{avl\_insert})
			\item Open and Parse Input File, yielding a statement graph.
			\item \texttt{statement = } start
			\item \texttt{while statement != NULL}:
			\begin{enumerate}
				\item Execute statement, yielding the next statement to execute.
				\item \texttt{statement = } next statement
			\end{enumerate}
			\item Print all variables in scope
			\item Free statement graph
			\item Free scopes
			\item Free command-line flags
		}
		\end{enumerate}
	}
	
	\section{Implementation/Tools Used}
	{
		\begin{itemize}
		{
			\item Use Bison and Flex for generating the parser.
			\item Use GMP for integer computation.
			\item Use Reference Counting for determining when to call
				\texttt{free} on \texttt{malloc}-ed memory.
			\item Use Make and GCC for building and compiling the program. Make
				is also used for building GMP if not already installed in
				the system.
			\item \texttt{mpz\_add}, \texttt{mpz\_sub}, \texttt{mpz\_mul}, and
				\texttt{mpz\_fdiv\_q} are the functions in the GMP library for
				doing additiona, subtraction, multiplication and division
				respectively.
			\item \texttt{libavl} can be used to create a binary tree to map
				variable names to values.
				(\texttt{avl\_alloc\_tree} \& \texttt{avl\_search}).
		}
		\end{itemize}
	}
	
	\section{Testing/Samples}
	{
		\texttt{\$ ./gen/pl -d `x = 3' -d `y = 5' ./examples/addition.txt}\\
		\texttt{x = 3, y = 5, z = 8}\\
		\texttt{\$ ./gen/pl -d `x = 9' -d `y = 3' ./examples/multiplication.txt}\\
		\texttt{x = 9, y = 3, z = 27}\\
		\texttt{\$ ./gen/pl -d `b = 3' -d `e = 5' ./examples/exponential.txt}\\
		\texttt{b = 3, e = 5, z = 243}\\
		\texttt{\$ ./gen/pl -d `x = 3' -d `y = 5' ./examples/arithmetic.txt}\\
		\texttt{a = 8, b = 0, c = 15, d = 0, e = 0, f = 1, g = 0, h = 0, i = 1, j = 1, k = 1, l = 1, m = 0, x = 3, y = 5}\\
		\texttt{\$ ./gen/pl -d `x = 3'./examples/fibonacci.txt}\\
		\texttt{a = 2, b = 3, c = 3, x = 0}\\
		\texttt{\$ ./gen/pl -d `a = 3' -d `b = 5' -d `c = 2' ./examples/sort-3.txt}\\
		\texttt{a = 3, b = 5, c = 2, switch = 1, x = 2, y = 3, z = 5}\\
	}
	
	\section{Goals Reached}
	{
		\begin{enumerate}
			\item rearrange grammar such that \texttt{bison} \& \texttt{flex}
			can generate a parser without errors. (done)
			
			\item use the reduction rules in generated parser to generate
			expression trees and statement graphs. (done)
			
			\item implement main computation loop and expression evaluation code
			using GMP for integer computation. (done)
			
			\item Ensure that at the execution of the program there are no
				memory leaks and that there are no valgrind errors. (done)
				
			\item Ensure that any input file that follows the above grammar
			could be executed by the program. This goal was not met for two reasons:
			\begin{enumerate}
				\item Programs with unreachable statements would result in an
				error. This is due to the fact that as the parses reads the input
				file, statement structs are allocated with references
				to the statements that would be transitioned to. For instance,
				an \texttt{if} statement has a reference to the first statement
				of the true case, as well as a reference to the first statement
				of the false case. These are the two paths forward in execution.
				Currently, the program throws an error if it cannot connect the
				whole program together in one long chain of references because
				as of right now, the program uses the chain of references to free
				the allocated statements at the end of execution and by allowing
				for multiple disconnected chains we would no longer be able to
				(conveniently) ensure no memory leaks.
				
				\item GOTOs can only go upwards in the file. This is due to the
				fact that according to the language specifcation, labels can be
				reassigned, and that a GOTO should transition to the statement
				most recently assigned to the given label at the time of parsing
				the GOTO. This behaviour does function but made it difficult to
				also implement the ability for GOTOs to refer to labels
				yet-to-be-declared.
			\end{enumerate}
		\end{enumerate}
	}
	
	\section{References}
	{
		\begin{itemize}
			\item https://gmplib.org/
			\item https://www.gnu.org/software/make/manual/html\_node/Automatic-Variables.html
			\item Theory of Computation by Walter S. Brainerd, page 38.
		\end{itemize}
	}
	
}
\end{document}






















